%%%%%%%%%%%%%%%%%%%%%%%%%%%%%%%%%%%%%%%%%%%%%%%%%%%%%%%%%%%%%%%%%%%%%%%%%%%%%%
\section{Introduction}
\label{sec:intro}
%%%%%%%%%%%%%%%%%%%%%%%%%%%%%%%%%%%%%%%%%%%%%%%%%%%%%%%%%%%%%%%%%%%%%%%%%%%%%%
%=============================================================================
\subsection{The MoEDAL experiment}
\label{sec:intromoedal}
%=============================================================================
%The \ac{Monopole and Exotics Detector at the LHC~\cite{MoEDAL2009}
The \ac{MoEDAL}~\cite{MoEDAL2009}
is the latest addition to a long line of experiments that have searched for
Dirac's hypothesised magnetic monopole~\cite{Dirac1931}.
%
Based at the Large Hadron Collider's Interaction Point 8 (IP8),
it is housed in the same experimental cavern as
the \acs{LHCb} experiment~\cite{LHCb2008}.
%
Like all experiments at the \acs{LHC},
meeting the demands of its physics programme~\cite{MoEDAL2014}
requires significant computing power.
Simulations of the detector need to be run,
data from the subdetectors needs to be stored and processed,
and analyses combining the two need to be performed.
%
The \ac{WLCG}~\cite{WLCG2005} was
designed to meet these needs (and more) of the \ac{LHC}
experiments, and thanks to the
United Kingdom's GridPP Collaboration~\cite{gridpp2006,gridpp2009}
\ac{MoEDAL} has access to the resources it provides.
%
This document aims to provide a guide for users involved
with the \ac{MoEDAL} experiment to harness the computing resources
offered by the Grid for the benefit of the
\ac{MoEDAL} Collaboration.

%=============================================================================
\subsection{For whom is this guide written?}
\label{sec:forwhom}
%=============================================================================
This guide is primarily aimed at people working directly on
the \ac{MoEDAL} experiment who wish to use Grid computing resources
for their work. You could be:

\begin{itemize}
\item a researcher from a \ac{MoEDAL} member institution;
\item a postgraduate student completing a PhD on \ac{MoEDAL};
\item an undergraduate student working on a project supervised by 
a \ac{MoEDAL} researcher;
\item a \ac{MoEDAL}/\acs{CERN} summer student;
\item a school student working with the \acf{IRIS}.
\end{itemize}

This user guide isn't really aimed at:

\begin{itemize}
\item users with no association with the \ac{MoEDAL} experiment, or;
\item readers with little or no computing experience.
\end{itemize}

%_____________________________________________________________________________
\begin{tcolorbox}[title=Computing knowledge]
\emph{While every effort has been made to cover as many
bases as possible, some computing knowledge is
assumed. You can read more about what you might need to know in
Section~\ref{sec:prerequisites}}.
\end{tcolorbox}
%_____________________________________________________________________________

\newpage

%=============================================================================
\subsection{Overview of the guide}
\label{sec:overview}
%=============================================================================
Section~\ref{sec:bwb} covers everything you need to know before
attempting to get started on the Grid with MoEDAL, including
prerequisites (Section~\ref{sec:prerequisites}),
conventions used in this guide (Section~\ref{sec:conventions}),
how to get help (Section~\ref{sec:help}),
and
how this document relates to the GridPP UserGuide (Section~\ref{sec:gridppuserguide}).
%
Section~\ref{sec:grid} describes how MoEDAL integrates with the Grid
infrastructure, looking at the MoEDAL Virtual Organisation (Section~\ref{sec:vo}),
how MoEDAL data is stored on the Grid (Section~\ref{sec:data}),
and
how MoEDAL software is distributed on the Grid with the
MoEDAL CernVM-FS repository (Section~\ref{sec:cvmfs}).
%
You'll find out how to run MoEDAL simulations in
Section~\ref{sec:sim},
which covers setup (Section~\ref{sec:moedalgridsetup}),
local running (Section~\ref{sec:moedallocalrunning}),
and
grid running (Section~\ref{sec:moedalgridrunning}).
%
After a brief look at the next steps to take (Section~\ref{sec:next}),
references, acronymns, and acknowledgements are provided in
Sections~\ref{sec:references}, \ref{sec:acronyms}, and
\ref{sec:ack} respectively.
