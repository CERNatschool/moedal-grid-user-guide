%=============================================================================
\subsection{Local running with Ganga}
\label{sec:moedallocalrunning}
%=============================================================================
First, go to the \code{ganga} directory within the \ac{MoEDAL} simulation script
repository, and then start Ganga:

\begin{lstlisting}[gobble=0,numbers=none,language=bash]
$ cd $WORKING_DIR/moedal-run-simulations/ganga
$ source /cvmfs/ganga.cern.ch/runGanga.sh 
\end{lstlisting}

You'll then see lots of output telling you about how Ganga has started,
including the location of your Ganga configuration file \texttt{.gangarc}.
Make a note of this.
The prompt will now change to an interactive iPython prompt.
It is from here that you can submit commands to Ganga.
We have prepared a script that submits a MoEDAL simulation
job to your local system. The beauty of Ganga is that it treats
local, batch and grid jobs in the same way, so all it takes is a small
tweak of your config file to change from local to grid running.
The script creates, configures, and submits
a small (four events) LHE simulation to your local system.
You can run it and submit the job with:

\begin{lstlisting}[gobble=0,numbers=none,language=bash]
Ganga In [x]: execfile(‘make_local_job.py’) 
Ganga In [x]: j.submit()
\end{lstlisting}

\begin{hintbox}{The Ganga command prompt}
\emph{Note: the small \texttt{x} will actually be the number of the command
you've typed into the Ganga prompt, and will keep incrementing
by one with every command...}
\end{hintbox}

You should then be informed that your job has been submitted via the
Ganga output.
You can check the status of your jobs with the following Ganga command:

\begin{lstlisting}[gobble=0,numbers=none,language=bash]
Ganga In [x]: jobs
\end{lstlisting}

Once your job has finished running (showing as \texttt{Completed} in the jobs table),
you can get the output directory using:

\begin{lstlisting}[gobble=0,numbers=none,language=bash]
Ganga In [x]: j = jobs(X) # where X is the job number in the table.
Ganga In [x]: j.outputdir
\end{lstlisting}

You can then inspect the contents of this directory
(including \texttt{stdout} and \texttt{stderr}) as usual.

You can find more information about using Ganga here:

\href{http://ganga.readthedocs.io}{http://ganga.readthedocs.io}

\begin{hintbox}{Aliasing Ganga}
\emph{Add an alias to your \texttt{~/.bashrc} file
to save typing the whole command every time:}

\code{alias run\_ganga\_cvmfs='source /cvmfs/ganga.cern.ch/runGanga.sh'}
\end{hintbox}

