%%%%%%%%%%%%%%%%%%%%%%%%%%%%%%%%%%%%%%%%%%%%%%%%%%%%%%%%%%%%%%%%%%%%%%%%%%%%%%
\section{Running MoEDAL simulations}
\label{sec:sim}
%%%%%%%%%%%%%%%%%%%%%%%%%%%%%%%%%%%%%%%%%%%%%%%%%%%%%%%%%%%%%%%%%%%%%%%%%%%%%%
We'll now look at running a simulation of the \ac{MoEDAL} experiment.
Specifically, we'll be running the simulation of
monopole-antimonopole production in 13 TeV proton-proton collisions
at Interaction Point 8, i.e. the \ac{LHCb} cavern.
We will use the \ac{LHCb} and \ac{MoEDAL} software frameworks to 
simulate the passage of the resulting (anti)monopole through the
\ac{MoEDAL} detector material to see how the various subdetectors perform.

\begin{infobox}{Analysing the results}
\emph{Note that we will not cover the analysis of the simulation results
here -- this is beyond the scope of this guide,
as (generally speaking) once you have the output files from the Grid
you don't need the Grid to do the analysis.}
\end{infobox}

We'll actually run the simulation on your local machine before running it
on the Grid.  This is actually a good thing to do generally,
as it will let you quickly debug any problems before running your code
on a remote system. Fortunately, switching between local running
and Grid running is simple thanks to the Ganga \ac{UI}\footnote{%
See \href{http://ganga.web.cern.ch}{http://ganga.web.cern.ch}}.

%=============================================================================
\subsection{Setting up}
\label{sec:moedalgridsetup}
%=============================================================================
Firstly, you'll need to clone the repository from the \ac{CERN}
GitLab\footnote{See \href{http://gitlab.cern.ch}{http://gitlab.cern.ch}}
system. Log on from your \ac{CVMFS}-enabled machine, create a working
directory in your user space (which we will call \code{\$WORKING\_DIR})
and check out the code as follows:

\begin{lstlisting}[gobble=0,numbers=none,language=bash]
$ cd $WORKING_DIR # wherever you want to run your simulations from.
$ git clone ssh://git@gitlab.cern.ch:7999/moedal/moedal-run-simulations.git
$ cd moedal-run-simulations
\end{lstlisting}

\begin{warningbox}{SSH keys}
\emph{This repository cloning method assumes you have your \ac{SSH} key in your
machine's \texttt{~/.ssh} directory, and that you have uploaded the
public key to the CERN GitLab instance, as described in
Section~\ref{sec:prerequisites}.}
\end{warningbox}

\begin{warningbox}{Port unblocking}
\emph{Note that the CERN GitLab instance is accessed via port 7999 -- make
sure this is unblocked in any of your system's firewalls.}
\end{warningbox}


\clearpage

%=============================================================================
\subsection{Local running with Ganga}
\label{sec:moedallocalrunning}
%=============================================================================
First, go to the \code{ganga} directory within the \ac{MoEDAL} simulation script
repository, and then start Ganga:

\begin{lstlisting}[gobble=0,numbers=none,language=bash]
$ cd $WORKING_DIR/moedal-run-simulations/ganga
$ source /cvmfs/ganga.cern.ch/runGanga.sh 
\end{lstlisting}

You'll then see lots of output telling you about how Ganga has started,
including the location of your Ganga configuration file \texttt{.gangarc}.
Make a note of this.
The prompt will now change to an interactive iPython prompt.
It is from here that you can submit commands to Ganga.
We have prepared a script that submits a MoEDAL simulation
job to your local system. The beauty of Ganga is that it treats
local, batch and grid jobs in the same way, so all it takes is a small
tweak of your config file to change from local to grid running.
The script creates, configures, and submits
a small (four events) LHE simulation to your local system.
You can run it and submit the job with:

\begin{lstlisting}[gobble=0,numbers=none,language=bash]
Ganga In [x]: execfile(‘make_local_job.py’) 
Ganga In [x]: j.submit()
\end{lstlisting}

\begin{hintbox}{The Ganga command prompt}
\emph{Note: the small \texttt{x} will actually be the number of the command
you've typed into the Ganga prompt, and will keep incrementing
by one with every command...}
\end{hintbox}

You should then be informed that your job has been submitted via the
Ganga output.
You can check the status of your jobs with the following Ganga command:

\begin{lstlisting}[gobble=0,numbers=none,language=bash]
Ganga In [x]: jobs
\end{lstlisting}

Once your job has finished running (showing as \texttt{Completed} in the jobs table),
you can get the output directory using:

\begin{lstlisting}[gobble=0,numbers=none,language=bash]
Ganga In [x]: j = jobs(X) # where X is the job number in the table.
Ganga In [x]: j.outputdir
\end{lstlisting}

You can then inspect the contents of this directory
(including \texttt{stdout} and \texttt{stderr}) as usual.

You can find more information about using Ganga here:

\href{http://ganga.readthedocs.io}{http://ganga.readthedocs.io}

\begin{hintbox}{Aliasing Ganga}
\emph{Add an alias to your \texttt{~/.bashrc} file
to save typing the whole command every time:}

\code{alias run\_ganga\_cvmfs='source /cvmfs/ganga.cern.ch/runGanga.sh'}
\end{hintbox}



\clearpage

%=============================================================================
\subsection{Grid running with Ganga}
\label{sec:moedalgridrunning}
%=============================================================================
\acs{DIRAC}~\cite{DIRAC2010}\footnote{%
See \href{http://diracgrid.org/}{http://diracgrid.org}} is a software
framework for distributed
computing that the GridPP Collaboration uses to manage
jobs and data on the Grid using their own
GridPP \ac{DIRAC} instance~\cite{GRIDPPDIRAC2015a,GRIDPPDIRAC2015b} --
so we're going to use it too.
To get started, we firstly need to change a few settings in our
Ganga configuration file to reflect the fact we're running
with the \ac{MoEDAL} \ac{VO}. Specifically, in your
\texttt{~/.gangarc} file, find and set the following variables:
 
\begin{lstlisting}[gobble=0,numbers=none,language=bash]
[defaults_DiracProxy]
...
group = vo.moedal.org_user
...
[DIRAC]
...
DiracLFNBase = /vo.moedal.org/user/a/ada.lovelace
...
\end{lstlisting}

(Except use your own \ac{DFC} user space, of course!)

Then we initialise \ac{DIRAC} and generate a Grid proxy as usual:

\begin{lstlisting}[gobble=0,numbers=none,language=bash]
$ source /cvmfs/ganga.cern.ch/dirac_ui/bashrc
$ dirac-proxy-init -g vo.moedal.org_user -M
Generating proxy... 
Enter Certificate password:
[...output...]
\end{lstlisting}

Remember how we said switching to Grid running with Ganga was trivial?
Well, we just have to use a slightly script to create the Grid-ready job
and then submit it as before:

\begin{lstlisting}[gobble=0,numbers=none,language=bash]
Ganga In [x]: execfile('make_dirac_job.py') 
Ganga In [x]: j.submit()
\end{lstlisting}

And that's it!
You can check on the status of your job in the GridPP DIRAC
system by accessing the GridPP DIRAC page here:

\href{https://dirac.gridpp.ac.uk}{https://dirac.gridpp.ac.uk}

With a browser that has your grid certificate installed.
You should select \term{vo.moedal.org\_user} from the bottom-right
drop down menu, and then \term{Job Monitor} from the \term{Jobs} drop-down menu
in the top-left corner.

\begin{warningbox}{Submitting jobs to the Grid}
\emph{Submitting jobs to the grid can take some time.
You might want to make a cup of tea or do something else while waiting!}
\end{warningbox}

%If all has gone well,
%you should now be able to retrieve the job output as per the
%instructions in the GridPP UserGuide.
%
%\begin{hintbox}{Getting help with errors}
%\emph{If the job fails, or you get errors, remember the help
%contacts available in the GridPP UserGuide!}
%\end{hintbox}


