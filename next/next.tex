%%%%%%%%%%%%%%%%%%%%%%%%%%%%%%%%%%%%%%%%%%%%%%%%%%%%%%%%%%%%%%%%%%%%%%%%%%%%%%
\section{What's next?}
\label{sec:next}
%%%%%%%%%%%%%%%%%%%%%%%%%%%%%%%%%%%%%%%%%%%%%%%%%%%%%%%%%%%%%%%%%%%%%%%%%%%%%%
Now you can submit jobs successfully, you can start to play
around with the parameters in \texttt{make\_dirac\_job.py}
to produce different simulation runs.
For example, you could use loops with the following variables to submit
many jobs at once:

\begin{lstlisting}[gobble=0,numbers=none,language=bash]
## The spin of the magnetic monopole.
monopole_spins = ['SpinZero', 'SpinHalf']

## A dictionary of monopole masses and corresponding LHE run name.
monopole_masses = { \
   200:'run_01', \
   500:'run_02', \
  1000:'run_03', \
  1500:'run_04', \
  2000:'run_05', \
  2500:'run_06', \
  3000:'run_07', \
  4000:'run_08', \
  5000:'run_09', \
  6000:'run_10'  \
}

## A dictionary of magnetic charges to corresponding LHE filenames.
monopole_magnetic_charges = { \
  1:'q10', \
  2:'q20', \
  3:'q30', \
  4:'q40', \
  5:'q50', \
  6:'q60'
}

## A dictionary of geometry names to geometry files.
geometries = { \
  'default':'geometry_default.db', \
  'maximal':'geometry_maximal.db', \
  'minimal':'geometry_minimal.db'
} 
\end{lstlisting}

%\begin{hintbox}{Getting Pythonic}
%\emph{You can use these dictionaries and loops to generate and
%submit many jobs at once.}
%\end{hintbox}

So there we go -- having followed this guide,
you should be able to submit MoEDAL
simulation jobs using Ganga and GridPP DIRAC.
You should also be able to develop new job configurations
for new physics scenarios. These should be added to
the GitLab repository here:

\href{https://gitlab.cern.ch/moedal/moedal-run-simulations}{https://gitlab.cern.ch/moedal/moedal-run-simulations}

Good luck!
