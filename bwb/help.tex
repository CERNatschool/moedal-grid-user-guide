%=============================================================================
\subsection{Getting help}
\label{sec:help}
%=============================================================================
There are many ways of getting help and support if you run into problems
while working with MoEDAL on the Grid.
If you don't happen
to have a Grid expert in the office down the corridor, you can try the
methods described below.

%-----------------------------------------------------------------------------
\subsubsection{Googling the error}
\label{sec:googleerror}
%-----------------------------------------------------------------------------
We can't possibly account for every error a user might encounter when
working through this guide, so on encountering a problem your
first port of call should be sticking the error message into your Search
Engine of Choice.

\begin{hintbox}{Let Me Google That For You}
\emph{This is actually a pretty good approach to software development in
general. Thanks to vibrant, enthusastic communities like those at
StackExchange many common computing gotchas have been documented and
solved on the World Wide Web - so it's always worth checking!}
\end{hintbox}

%-----------------------------------------------------------------------------
\subsubsection{The MoEDAL mailing lists}
\label{sec:mailinghelp}
%-----------------------------------------------------------------------------
The mailing lists listed in Table~\ref{tab:moedal-help-mailing-lists}
can provide support from both the MoEDAL and GridPP Collaborations.
%______________________________________________________________________________
\begin{table}[ht]
\small
\caption{\label{tab:moedal-help-mailing-lists}MoEDAL-related support
mailing lists.
To join them, access the corresponding CERN e-Group
or JISC page and submit a request to join.}
\lineup
\begin{tabular}{@{}lll}
\br
\centre{1}{$\quad$List        $\quad$} &
\centre{1}{$\quad$Description $\quad$} &
\centre{1}{$\quad$Provider    $\quad$} \\
\mr
%%_____________________________________________________________________________
\code{MoEDAL-Software@cern.ch}       & For MoEDAL software support.      & CERN \\
\code{MoEDAL-gridsupport@cern.ch}    & For MoEDAL-specific Grid support. & CERN \\
\code{GRIDPP-SUPPORT@jiscmail.ac.uk} & For support from GridPP.          & \href{https://www.jiscmail.ac.uk/cgi-bin/webadmin?SUBED1=GRIDPP-SUPPORT\&A=1}{JISCMail} \\
%_____________________________________________________________________________
\br
\end{tabular}
\end{table}
%______________________________________________________________________________


%-----------------------------------------------------------------------------
\subsubsection{GridPP-specific issues}
\label{sec:gridpphelp}
%-----------------------------------------------------------------------------
If you have a problem relating specifically to the software or
infrastructure supported by GridPP (e.g. \acs{DIRAC}, Ganga, \acs{CVMFS}, etc.)
the easiest way to report problems
is by raising an \textbf{issue} on the
\emph{GridPP UserGuide}'s
\href{http://github.com/GridPP/user-guides}{GitHub repository}.
Simply log in to GitHub, visit the \emph{UserGuide}
\href{https://github.com/gridpp/user-guides/issues}{issues page} and
click on the \href{https://github.com/gridpp/user-guides/issues/new}{New
issue} button.

Provide as much information as you can when raising an issue. You can
also use the MarkDown format to create hyperlinks and add formatting to
your issue.

Don't forget to \textbf{Watch} the repository too. You can do this by going to
the repository, signing in with your GitHub account, and clicking on the
Watch button at the top-right of the page. You'll then be kept
up-to-date with issues and new versions as the GridPP infrastructure
evolves over time.
