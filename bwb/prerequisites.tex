%=============================================================================
\subsection{Prerequisites}
\label{sec:prerequisites}
%=============================================================================
If you're working on the \ac{MoEDAL} experiment,
it's very likely that you will have most of the things listed below.
But there's no harm in checking!

\begin{itemize}
\tightlist
\item
  \textbf{A valid email address}: This sounds kind of obvious - and who
  \emph{doesn't} have an email address these days? - but you'll need a
  valid email address from which you can send and receive emails.
\end{itemize}

%_____________________________________________________________________________
\begin{tcolorbox}[title=Institutional email accounts]
\emph{You should use the email account
given to you by the school or university associated with the
\ac{MoEDAL} Collaboration, as this
will make life a little easier when it comes to granting you access to
Grid resources. Only use your \acs{CERN} email address (if you have one)
if it is associated with your Grid certificate.}
\end{tcolorbox}
%_____________________________________________________________________________

\begin{itemize}
\tightlist
\item
  \textbf{A GitHub account}: Much of
  the software used by the GridPP Collaboration
  and associated partner organisations is hosted on their
  public GitHub repositories - see, for example, the
  \href{http://github.com/gridpp}{the GridPP GitHub repository}\footnote{
\href{http://github.com/gridpp}{http://github.com/gridpp}
} You can sign up for a free GitHub account on \href{http://github.com}{the
  GitHub website}.
\item
  \textbf{Experience with the command line}: The command line allows you
  to type instructions into your computer in order to get it to do
  things for you, rather than relying on clicking on icons, buttons, and
  other graphical elements of a software package. In his guide,
  \href{http://www.learnenough.com/command-line-tutorial}{Learn Enough
  Command Line to be Dangerous}\footnote{
\href{http://www.learnenough.com/command-line-tutorial}{http://www.learnenough.com/command-line-tutorial}
}, Michael Hartl uses a nice analogy with
  magic. While it is \emph{technically} possible to use the Grid without
  using the command line (using, for example, a web browser to access
  specific Grid systems), using the command line is infinitely easier
  and gives you much, much more flexibility. Hartl's
  \href{http://www.learnenough.com/command-line-tutorial}{tutorial}, is
  well worth following if you've not used it before (or even if you
  have!).
\item
  \textbf{A text editor}: We'll be writing scripts - series of commands
  to be executed one after the other - and for this you'll need a text
  editor of some description. Emacs, Vim, Vi - whatever you feel most
  comfortable with. Vim, for example, allows you to edit text from the
  command line.
\item
  \textbf{Programming with Python}: Once we start getting fancy with the
  \ac{MoEDAL} software and the
  Grid, we're going to use the Python programming language
  to do a lot of the work for us. 
  As such, some familiarity with Python will be handy. There are plenty of
  (free!) online tutorials available that can get you started. We'll
  provide plenty of examples too, so don't panic.
\item
  \textbf{A Scientific Linux 6 command line with CVMFS access}: This
  will either be provided by your institution or
  via a \href{../gridpp-cernvm/gridpp-cernvm.html}{GridPP CernVM}, a
  \acf{VM} made by \href{http://cern.home}{CERN} that you can run
  yourself. The \href{https://cernvm.cern.ch/}{CernVM-File System},
  a.k.a. CernVM-FS or \acs{CVMFS}, gives you (and any grid node, for that
  matter) instant access to all sorts of software \emph{without having
  to install anything}. So it's worth sorting out!
  You can find instructions for creating a
  GridPP CernVM  \href{http://www.gridpp.ac.uk/userguide}{the GridPP UserGuide}.
\item
  \textbf{Access to a CERN GitLab account} (\acs{CERN} users only):
\acs{CERN} has its own Git repository service, and every \acs{CERN} user automatically
has an account on it. You'll need to log on and add an \acs{SSH} key to use
it though - see [this page](https://gitlab.cern.ch/help/ssh/README.md) for more
information.
\item
  \textbf{An SSH key}:
\acf{SSH} keys are a way to identify trusted computers without involving passwords.
You can generate an \acs{SSH} key and add the public key to your GitHub account
by following the procedures outlined in this
\href%
{https://help.github.com/articles/generating-an-ssh-key/}%
{this guide from GitHub}\footnote{%
See \href{https://help.github.com/articles/generating-an-ssh-key/}{https://help.github.com/articles/generating-an-ssh-key/}
}.
\end{itemize}

Got/done all of that? Good. Now let's look at the
conventions used in this guide.
