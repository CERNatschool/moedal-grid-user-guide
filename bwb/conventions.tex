%=============================================================================
\subsection{Conventions used in this guide}
\label{sec:conventions}
%=============================================================================
The conventions used in this guide are, by and large,
self-explanatory. Here we'll look at a few that might not be.

%-----------------------------------------------------------------------------
\subsubsection{The command line}
\label{sec:the-command-line}
%-----------------------------------------------------------------------------

Following \href{https://www.railstutorial.org}{Hartl}, we'll present
command line examples using a Unix-style command line prompt (a dollar
sign), as follows:

%CODING SNIPPET
%
\begin{lstlisting}[gobble=0,numbers=none,language=bash]
$ echo "Hello, MoEDAL!"
Hello, MoEDAL!
\end{lstlisting} 

i.e.~you type what follows the dollar sign, and hopefully see the same
(dollar sign-less) output in your terminal.

%_____________________________________________________________________________
\begin{warningbox}{Comparing the output}
\emph{Computer systems are always going to vary from machine to machine, so
you may not see exactly the same output from a given command.
If something doesn't look right, 
more often than not a combination of common sense and
Googling the output should confirm if you're on the right track.}
\end{warningbox}
%_____________________________________________________________________________

Where possible we'll use bash environment variables to account for
differences in working directories. However, if a particularly
user-specific input is needed we'll use square brackets to denote parts
of the command that require input specific to your circumstances. For
example, when setting your working directory environment variable
\texttt{WORKING\_DIR}, we'd write this:

\begin{lstlisting}[gobble=0,numbers=none,language=bash]
$ export WORKING_DIR=[Your working directory.]
$ echo $WORKING_DIR
[The value of $WORKING_DIR, hopefully your working directory.]
\end{lstlisting} 

which would actually be completed using:

\begin{lstlisting}[gobble=0,numbers=none,language=bash]
$ export WORKING_DIR=/home/lmeitner/moedal-stuff/
$ echo $WORKING_DIR
/home/lmeitner/moedal-stuff/
\end{lstlisting} 

%-----------------------------------------------------------------------------
\subsubsection{Code listings}
\label{sec:code-listings}
%-----------------------------------------------------------------------------
Generally speaking, we have tried to avoid listing large swathes of code
in this guide - that's what GitHub is for. From
time-to-time it may be useful to include a code snippet like the
following:

\begin{lstlisting}[gobble=0,numbers=none,language=python]
#!/usr/bin/env python
print("* This works!")
\end{lstlisting}

Following \href{https://www.railstutorial.org}{Hartl}, we will use
vertical dots to represent code omitted for the sake of brevity:

\begin{lstlisting}[gobble=0,numbers=none,language=python]
#!/usr/bin/env python

class GridJob:
    .
    .
    .
    def submit(self, id):
        self.id = id
        .
        .
        .
\end{lstlisting}

These dots should not be copied into your code.

%-----------------------------------------------------------------------------
\subsubsection{Hints, warnings, and information}
\label{sec:hints-warnings-and-information}
%-----------------------------------------------------------------------------
This guide, like many instructional handbooks, uses
little pop-out boxes to highlight important points throughout the text.

\begin{hintbox}{Hint}
\emph{This is a hint. Hint boxes are used for pointing out things that might
be useful while carrying out the task being described (particularly
where we have received user feedback on a given step!).}
\end{hintbox}

\begin{warningbox}{Warning!}
\emph{This is a warning. These are used to flag up potential pitfalls or
issues you may need to be aware of to avoid making mistakes or doing
Something Bad.}
\end{warningbox}

\begin{infobox}{Point of information}
\emph{This is a point of information. These boxes will generally present
things that may not directly relate to the topic being discussed but are
nonetheless interesting.}
\end{infobox}

%Thanks to Font Awesome for the icons! They're, um, awesome.
%
%\subsection{Checklists}\label{checklists}
%
%Once you've waded through the waffle associated with a given section,
%you'll be presented with a \textbf{checklist} section that will give you
%a simple, bullet-pointed list of the things you should be able to do
%once you've read that section. You should go through these to make sure
%you have done them and, more importantly, understood them. If not,
%re-read the section. Alternatively, you could plough on and try the
%tests - see below - to see if it makes more sense when you try to
%actually do something based on what you've just read.
%
%\subsection{Testing}\label{testing}
%
%All well-written, well-packaged code should come complete with
%\textbf{unit tests}; scripts or bits of code that can be run to test
%whether one's software is working as expected (especially during
%development as changes are made and new versions are produced). We can
%try to emulate this approach by trying to test the success of each of
%the steps taken while following the instructions presented in the
%\emph{UserGuide}. At the end of each section you will therefore find a
%\textbf{Testing} page that will present a number of tasks or tests for
%you to complete to verify that you have followed the \emph{UserGuide}.
%As a rule you should not proceed to the next section until you have
%passed all of these tests.
%
%If you're struggling, there are plenty of ways to get help and support.
%We'll find out more about these in the \href{getting-help.html}{next
%section}.
%
