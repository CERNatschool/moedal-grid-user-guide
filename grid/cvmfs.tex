%=============================================================================
\subsection{The MoEDAL CVMFS repository}
\label{sec:cvmfs}
%=============================================================================
The CernVM File System -- a.k.a. CernVM-FS or \acs{CVMFS} -- is used to
distribute software on the Grid.  You can read more about it
in~\cite{CVMFS2015}.  It is part of the CernVM project~\cite{CernVM2015}.
You'll have read about it in the GridPP UserGuide too, but in essence it means
\ac{MoEDAL}'s software will be instantly available on any
\ac{CVMFS}-enabled computing node you use.

%-----------------------------------------------------------------------------
\subsubsection{Finding the MoEDAL CVMFS repository}
\label{sec:moedalcvmfsfinding}
%-----------------------------------------------------------------------------

MoEDAL's \ac{CVMFS} repository can be found here:

\texttt{/cvmfs/moedal.cern.ch/}

So, for example, you can find MoEDAL's Gauss software by logging on to
any computing node with the MoEDAL \ac{CVMFS} repository enabled
(e.g. a GridPP CernVM) and typing:

\begin{lstlisting}[gobble=0,numbers=none,language=bash]
$ ls /cvmfs/moedal.cern.ch/Gauss
Gauss_v48r1  Geometry
\end{lstlisting}

Simulations of the MoEDAL experiment, based on \ac{LHCb}'s Gauss
software, are currently running with \texttt{v48r1}.
The custom MoEDAL Gauss modules used may be found here:

\begin{lstlisting}[gobble=0,numbers=none,language=bash]
$ ls /cvmfs/moedal.cern.ch/Gauss/Gauss_v48r1/Sim
Gauss  GaussMoEDAL  GaussMonopoles
\end{lstlisting}

\begin{infobox}{CVMFS and the LHCb software framework}
\emph{As you'll see from the next section, we actually use a lot
of \ac{LHCb}'s software in \ac{MoEDAL} simulations -- and it too
is available via \ac{CVMFS}!}
\end{infobox}

We also use \ac{CVMFS} to distribute the latest \ac{MoEDAL}
geometry files. The latest version of the MoEDAL geometry
at the time of writing, \texttt{2.0.0}, may be found here:

\begin{lstlisting}[gobble=0,numbers=none,language=bash]
ls /cvmfs/moedal.cern.ch/Gauss/Geometry/2-0-0/
geometry_default.db  geometry_maximal.db  geometry_minimal.db
\end{lstlisting}

As you'll see in the next section, these are used in the
simulations -- something that is made trivial by publishing them
on the \ac{CVMFS} repository.

\begin{warningbox}{Do not use CVMFS for data!}
\emph{Note that we do not use CVMFS for data. This is what the
\ac{DFC} is for. The only exception to this might be data used for
software testing and validation, but even then using Grid data
as part of your tests might not be a bad thing!}
\end{warningbox}

\clearpage


%-----------------------------------------------------------------------------
\subsubsection{Managing the MoEDAL CVMFS repository}
\label{sec:moedalcvmfsmanaging}
%-----------------------------------------------------------------------------

To add and remove software from the MoEDAL \ac{CVMFS} repository,
you will need to be a member of the
\texttt{LxCvmfs-moedal} \ac{CERN} e-Group\footnote{
Full title: Writers of \texttt{/cvmfs/moedal.cern.ch}}.
To access the repository, once you are confirmed as a member of the
\texttt{LxCvmfs-moedal} e-Group, log onto \ac{CERN}'s lxplus system
with your \ac{CERN} computing account:

\begin{lstlisting}[gobble=0,numbers=none,language=bash]
$ ssh -Y alovelace@lxplus.cern.ch
Password: 
* ********************************************************************
* Welcome to lxplus0048.cern.ch, SLC, 6.8
* Archive of news is available in /etc/motd-archive
* Reminder: You have agreed to comply with the CERN computing rules
* https://cern.ch/ComputingRules
* Puppet environment: production, Roger state: production
* Foreman hostgroup: lxplus/nodes/login
* LXPLUS Public Login Service
* ********************************************************************
\end{lstlisting}

\begin{warningbox}{Managing CVMFS repositories}
\emph{As you may have guessed, you'll need a CERN computing account
to manage the MoEDAL CVMFS repository. We'll assume that if you've
got as far as needed to manage the repository, this shouldn't be
a problem.}
\end{warningbox}

From there you'll be able to log into the MoEDAL \ac{CVMFS} repository:

\begin{lstlisting}[gobble=0,numbers=none,language=bash]
$ ssh -Y lovelace@cvmfs-moedal.cern.ch
\end{lstlisting}

After which you'll be presented with most of the instructions you need
to start a \ac{CVMFS} \term{transaction},
which you'll need to do to make changes to what's in the repository.
You should refer to the full instructions for managing
a \ac{CVMFS} repository on the CernVM TWiki page:

\href{https://twiki.cern.ch/twiki/bin/view/CvmFS/MaintainRepositories}{https://twiki.cern.ch/twiki/bin/view/CvmFS/MaintainRepositories}

Which also includes useful hints and tips on directory structure,
managing your software, and how to get the best out of the
\ac{CVMFS} system.
